%%%%%%%%%%%%%%%%%%%%%%%%%%%%%%%%%%%%%%%%%%%%%%%%%%%%%%%%%%%%%%%%%
%%% %
%%% % weiiszablon.tex
%%% % The Faculty of Electrical and Computer Engineering
%%% % Rzeszow University Of Technology diploma thesis Template
%%% % Szablon pracy dyplomowej Wydziału Elektrotechniki 
%%% % i Informatyki PRz
%%% % January, 2024
%%%%%%%%%%%%%%%%%%%%%%%%%%%%%%%%%%%%%%%%%%%%%%%%%%%%%%%%%%%%%%%%%

\documentclass[12pt,twoside]{article}

\usepackage{weiiszablon}

\author{Michał Bazan}

% np. EF-123456, EN-654321, ...
\studentID{EF-163881}

\title{Badanie algorytmów nawigacyjnych}
\titleEN{Temat pracy po angielsku}


%%% wybierz rodzaj pracy wpisując jeden z poniższych numerów: ...
% 1 = inżynierska	% BSc
% 2 = magisterska	% MSc
% 3 = doktorska		% PhD
% 4 = praca inżynierska
%%% na miejsce zera w linijce poniżej
\newcommand{\rodzajPracyNo}{2}


%%% promotor
\supervisor{dr inż. Dariusz Rzońca}
%% przykład: dr hab. inż. Józef Nowak, prof. PRz

%%% promotor ze stopniami naukowymi po angielsku
\supervisorEN{Dariusz Rzońca, dr. engineer}

\abstract{Praca koncentruje się na badaniu różnych algorytmów nawigacyjnych z wykorzystaniem zbudowanego robota mobilnego, który został skonstruowany zgodnie z procedurami ASPICE. Celem jest ocena dokładności, szybkości wyznaczania trasy oraz ogólnej wydajności tych algorytmów. Badania obejmują również wpływ rodzaju algorytmu odometrii na dokładność nawigacji. Analiza wyników pozwoli wyciągnąć wnioski dotyczące skuteczności i efektywności badanych algorytmów nawigacyjnych. Dzięki zastosowaniu standardów ASPICE zapewniona została wysoka jakość procesu budowy robota, co umożliwia rzetelne i wiarygodne badania nad jego funkcjonalnością i algorytmami nawigacyjnymi.}
\abstractEN{The work focuses on investigating different navigation algorithms using a built mobile robot that has been constructed according to ASPICE procedures. The aim is to evaluate the accuracy, routing speed and overall performance of these algorithms. The research also includes the influence of the type of odometry algorithm on navigation accuracy. Analysis of the results will allow conclusions to be drawn regarding the effectiveness and efficiency of the navigation algorithms under study. Through the use of ASPICE standards, the high quality of the robot's construction process is ensured, enabling reliable and credible research into its functionality and navigation algorithms.}

\keywords{Algorytmy nawigacyjne, Robot mobilny, Inżynieria, ASPICE, Badania}
\keywordsEN{Navigation algorithms, Mobile robot, Engineering, ASPICE, Research}


\begin{document}

% strona tytułowa
\maketitle

\blankpage

% spis treści
\tableofcontents

\clearpage
\blankpage


\section*{Wykaz symboli, oznaczeń i skrótów}


\section{Wstęp}
W dzisiejszych czasach, wraz z dynamicznym rozwojem technologii mobilnych, algorytmy nawigacyjne odgrywają kluczową rolę w różnorodnych aplikacjach, począwszy od systemów nawigacji GPS w samochodach po autonomiczne roboty poruszające się w różnych środowiskach. W niniejszej pracy skupimy się na badaniu algorytmów nawigacyjnych w kontekście ich dokładności, szybkości wyznaczania trasy oraz ogólnej wydajności, z wykorzystaniem zbudowanego robota mobilnego.

Celem pracy jest przede wszystkim wykonanie i zaprogramowanie robota mobilnego, który będzie wykorzystywany do testowania różnych algorytmów nawigacyjnych. Praca koncentruje się na badaniu wybranych algorytmów pod kątem ich dokładności, szybkości wyznaczania trasy oraz ogólnej wydajności w praktycznych warunkach. Ponadto, zbadana zostanie dokładność nawigacji w zależności od wykorzystanego algorytmu wyznaczania odometrii.

Zakres pracy obejmuje przede wszystkim konstrukcję i programowanie robota mobilnego, wybór, implementację i testowanie różnych algorytmów nawigacyjnych, oraz analizę wyników pod kątem z góry określonych kryteriów, takich jak dokładność, szybkość i wydajność. Praca będzie obejmować także badanie wpływu różnych czynników, takich jak rodzaj terenu czy charakterystyka środowiska, na działanie algorytmów nawigacyjnych oraz dokładność ich działania.

W niniejszym wstępie zdefiniowano główne cele oraz zakres pracy, który obejmuje badanie algorytmów nawigacyjnych w kontekście ich dokładności, szybkości wyznaczania trasy oraz ogólnej wydajności. Kolejne rozdziały będą skupiały się na szczegółowym opisie procesu realizacji pracy oraz analizie uzyskanych wyników.

\section{Inżynierska część projektu}

W tym rozdziale przedstawiono kompleksowy opis prac związanych z implementacją oraz funkcjonowaniem robota. Niniejszy rozdział stanowi szczegółowe omówienie trzech kluczowych elementów projektu, które skupiały się na budowie fizycznej robota zgodnie z procedurami ASPICE, implementacji oprogramowania w języku C++ z uwzględnieniem unit testów oraz wykorzystaniu środowiska ROS2 i technologii Docker do sterowania robotem. Dzięki zastosowaniu tych trzech elementów możliwe było zapewnienie nie tylko skutecznej implementacji samego robota, ale również jego oprogramowania oraz efektywnego zarządzania nim w czasie rzeczywistym.

W dalszej części pracy szczegółowo omówione zostaną poszczególne etapy realizacji każdego z tych komponentów, prezentując zarówno teoretyczne założenia, jak i praktyczne wyniki osiągnięte w ramach projektu.

\subsection{Budowa robota}

Pierwszym aspektem, który zostanie przedstawiony, jest proces budowy robota. Opisane zostaną tutaj szczegóły dotyczące wyboru komponentów, proces montażu oraz ewentualne modyfikacje wprowadzone w celu zapewnienia optymalnego działania.

\subsection{System operacyjny robota}

Ta sekcja skupia się na implementacji oprogramowania w języku C++, obejmującej zarówno projektowanie jak i programowanie robotycznych funkcji. 

\subsection{Sterowanie robotem}

Kolejnym istotnym zagadnieniem będzie omówienie wykorzystania ROS2 oraz technologii Docker w kontekście sterowania robotem. Przedstawione zostaną tutaj zarówno zalety jak i wyzwania związane z integracją tych narzędzi.

\section{Badane algorytmy}

Niniejszy rozdział skupia się na opisie wybranych algorytmów nawigacyjnych, które zostały zastosowane w projekcie robota. Głównym celem tego rozdziału jest zaprezentowanie różnorodności podejść algorytmicznych wykorzystywanych w obszarze nawigacji robotów oraz ich efektywności w kontekście konkretnego projektu. W trakcie analizy omówione zostaną zarówno klasyczne, dobrze znane algorytmy, jak i te bardziej zaawansowane, specjalnie dostosowane do specyfiki projektu.

\section{Badania}

Niniejszy rozdział skupia się na analizie oraz porównaniu wybranych algorytmów nawigacyjnych pod względem ich dokładności nawigacji oraz wydajności obliczeniowej. Zostanie również zbadane, w jaki sposób rodzaj wybranego algorytmu odometrii może wpływać na ogólną dokładność nawigacji robota.

Wydajność obliczeniowa jest kluczowym czynnikiem, który należy uwzględnić przy ocenie algorytmów nawigacyjnych. W kontekście systemów wbudowanych, szybkość działania algorytmów ma kluczowe znaczenie dla płynności i efektywności pracy robota. Badania wydajności obliczeniowej pozwalają ocenić, jakie są różnice w zużyciu zasobów systemowych dla poszczególnych algorytmów oraz jakie są potencjalne ograniczenia związane z ich implementacją.

Dokładność nawigacji stanowi istotny wskaźnik efektywności systemów robotycznych, szczególnie w kontekście ich zdolności do precyzyjnego poruszania się w różnorodnych środowiskach. Analiza dokładności nawigacji pozwala na lepsze zrozumienie, jak poszczególne algorytmy radzą sobie z interpretacją danych sensorycznych oraz wyznaczaniem optymalnych trajektorii w czasie rzeczywistym.

Dodatkowo, analizie poddany zostanie wpływ rodzaju wybranego algorytmu odometrii na dokładność nawigacji. Odometria, jako technika wykorzystująca dane z czujników wbudowanych w układ jezdny robota, pełni kluczową rolę w określaniu jego położenia i orientacji w przestrzeni. Analiza tego wpływu pozwoli na zrozumienie wpływu precyzyjności danych odometrii na skuteczność działania algorytmów nawigacyjnych.

Przeprowadzone badania oraz uzyskane wyniki pozwolą wyciągnąć wnioski dotyczące skuteczności i efektywności badanych algorytmów nawigacyjnych, co stanowi istotny krok w doskonaleniu systemów nawigacyjnych robotów.

\section{Podsumowanie i wnioski końcowe}



\section*{Załączniki}



\clearpage

\addcontentsline{toc}{section}{Literatura}

\begin{thebibliography}{4}
\bibitem{repo} https://github.com/DevxMike/master\_degree
%\bibitem{Jakubczyk1997} Jakubczyk T., Klette A.: Pomiary w akustyce. WNT, Warszawa 1997.
%\bibitem{Barski2011} Barski S.: Modele transmitancji. Elektronika praktyczna, nr 7/2011, str. 15-18.
%\bibitem{dokum} Czujnik S200. Dokumentacja techniczno-ruchowa. Lumel, Zielona Góra, 2001.
%\bibitem{Pawluk2001} Pawluk K.: Jak pisać teksty techniczne poprawnie, Wiadomości Elektrotechniczne, Nr 12, 2001, str. 513-515.
\end{thebibliography}

\clearpage

\makesummary

\end{document} 
