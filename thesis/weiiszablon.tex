%%%%%%%%%%%%%%%%%%%%%%%%%%%%%%%%%%%%%%%%%%%%%%%%%%%%%%%%%%%%%%%%%
%%% %
%%% % weiiszablon.tex
%%% % The Faculty of Electrical and Computer Engineering
%%% % Rzeszow University Of Technology diploma thesis Template
%%% % Szablon pracy dyplomowej Wydziału Elektrotechniki 
%%% % i Informatyki PRz
%%% % January, 2024
%%%%%%%%%%%%%%%%%%%%%%%%%%%%%%%%%%%%%%%%%%%%%%%%%%%%%%%%%%%%%%%%%

\documentclass[12pt,twoside]{article}

\usepackage{weiiszablon}

\author{Michał Bazan}

% np. EF-123456, EN-654321, ...
\studentID{EF-163881}

\title{Badanie algorytmów nawigacyjnych}
\titleEN{Temat pracy po angielsku}


%%% wybierz rodzaj pracy wpisując jeden z poniższych numerów: ...
% 1 = inżynierska	% BSc
% 2 = magisterska	% MSc
% 3 = doktorska		% PhD
% 4 = praca inżynierska
%%% na miejsce zera w linijce poniżej
\newcommand{\rodzajPracyNo}{2}


%%% promotor
\supervisor{dr inż. Dariusz Rzońca}
%% przykład: dr hab. inż. Józef Nowak, prof. PRz

%%% promotor ze stopniami naukowymi po angielsku
\supervisorEN{Dariusz Rzońca, dr. engineer}

\abstract{Praca koncentruje się na badaniu wybranych algorytmów nawigacyjnych oraz metod optymalizacji nastaw regulatorów PID z wykorzystaniem zbudowanego robota mobilnego, który został skonstruowany zgodnie z procedurami ASPICE. Celem jest ocena dokładności, szybkości wyznaczania trasy oraz ogólnej wydajności tych algorytmów. Badania obejmują również wpływ rodzaju algorytmu odometrii na dokładność nawigacji. Analiza wyników pozwoli wyciągnąć wnioski dotyczące skuteczności i efektywności badanych algorytmów nawigacyjnych. Dzięki zastosowaniu standardów ASPICE zapewniona została wysoka jakość procesu budowy robota, co umożliwia rzetelne i wiarygodne badania nad jego funkcjonalnością i algorytmami nawigacyjnymi.}
\abstractEN{
The work focuses on the study of selected navigation algorithms and methods for optimising PID controller settings using a built mobile robot that has been constructed according to ASPICE procedures. The aim is to evaluate the accuracy, routing speed and overall performance of these algorithms. The research also includes the influence of the type of odometry algorithm on navigation accuracy. Analysis of the results will allow conclusions to be drawn regarding the effectiveness and efficiency of the navigation algorithms studied. Through the use of ASPICE standards, the high quality of the robot construction process is ensured, enabling reliable and credible research into its functionality and navigation algorithms.}

\keywords{Algorytmy nawigacyjne, Robot mobilny, Inżynieria, ASPICE, Machine Learning}
\keywordsEN{Navigation algorithms, Mobile robot, Engineering, ASPICE, Machine Learning}


\begin{document}

% strona tytułowa
\maketitle

\blankpage

% spis treści
\tableofcontents

\clearpage
\blankpage


\section*{Wykaz symboli, oznaczeń i skrótów}


\section{Wstęp}
W dzisiejszych czasach, wraz z dynamicznym rozwojem technologii mobilnych, algorytmy nawigacyjne i uczenia maszynowego \cite{deepLearning} odgrywają kluczową rolę w różnorodnych aplikacjach, począwszy od systemów nawigacji w samochodach po autonomiczne roboty poruszające się w różnych środowiskach. Biorąc pod uwagę aktualność tych zagadnień i rosnące zapotrzebowanie, zdecydowano o przeprowadzeniu badań dotyczących systemów nawigacyjnych.
   
Celem tej pracy jest zbadanie heurystycznych metod \cite{genetics} optymalizazcji regulatorów PID oraz porównanie wybranych algorytmów pod kątem kryteriów takich jak dokładność wyznaczania trasy oraz wydajność obliczeniowa w różnych warunkach terenowych.  

Zakres pracy obejmuje dwie części:
\begin{enumerate}[label=\alph*), leftmargin=1.25cm]
	\item część inżynierska - wykonanie robota mobilnego, implementacja systemu wbudowanego oraz implementacja stacji operatorskiej opartej na systemie ROS2 \cite{ros}
	\item część badawcza - badanie algorytmów uczenia maszynowego do optymalizacji nastaw regulatorów PID oraz porównanie wybranych algorytmów nawigacyjnych. 
\end{enumerate}

Zastosowanie standardu ASPICE \cite{SPICE} zapewnia wysoką jakość procesu budowy robota oraz implementacji oprogramowania co pozwola na przeprowadzenie rzetelnych badań i wyciągnięcie wiarygodnych wniosków.

\section{Inżynierska część projektu}

W tym rozdziale przedstawiono kompleksowy opis prac związanych z implementacją oraz funkcjonowaniem robota. Niniejszy rozdział stanowi szczegółowe omówienie trzech kluczowych elementów projektu, które skupiały się na budowie fizycznej robota zgodnie z procedurami SPICE, implementacji oprogramowania w języku C++ z uwzględnieniem unit testów oraz wykorzystaniu środowiska ROS2 i technologii Docker do sterowania robotem. Dzięki zastosowaniu tych trzech elementów możliwe było zapewnienie nie tylko skutecznej implementacji samego robota, ale również jego oprogramowania oraz efektywnego zarządzania nim w czasie rzeczywistym.

W dalszej części pracy szczegółowo omówione zostaną poszczególne etapy realizacji każdego z tych komponentów, prezentując zarówno teoretyczne założenia, jak i praktyczne wyniki osiągnięte w ramach projektu.

\subsection{Budowa robota}

Pierwszym aspektem, który zostanie przedstawiony, jest proces budowy robota. Opisane zostaną tutaj szczegóły dotyczące wyboru komponentów, proces montażu oraz ewentualne modyfikacje wprowadzone w celu zapewnienia optymalnego działania.

\subsection{System operacyjny robota}

Ta sekcja skupia się na implementacji oprogramowania w języku C++, obejmującej zarówno projektowanie jak i programowanie robotycznych funkcji. 

\subsection{Sterowanie robotem}

Kolejnym istotnym zagadnieniem będzie omówienie wykorzystania ROS2 oraz technologii Docker w kontekście sterowania robotem. Przedstawione zostaną tutaj zarówno zalety jak i wyzwania związane z integracją tych narzędzi.

\section{Badane algorytmy}

\subsection{Algorytm genetyczny}
\subsection{Uczenie ze wzmocnieniem}
\subsection{Algorytmy nawigacyjne}


\section{Badania}
\subsection{Optymalizacja nastaw regulatora PID prędkości obrotowej}
\subsection{Optymalizacja nastaw regulatora położenia i orientacji}
\subsection{Porównanie algorytmów nawigacyjnych w terenie bez przeszkód}
\subsection{Porównanie algorytmów nawigacyjnych w terenie z przeszkodami}

\section{Podsumowanie i wnioski końcowe}


\section*{Załączniki}



\clearpage

\addcontentsline{toc}{section}{Literatura}

\begin{thebibliography}{6}
\bibitem{repo} https://github.com/DevxMike/master\_degree
\bibitem{ros} https://docs.ros.org/en/foxy/index.html
\bibitem{deepLearning} Francois Chollet: Deep Learning. Praca z językiem Python i biblioteką Keras. Helion 2019
\bibitem{RL} Paweł Cichosz: Systemy uczące się. WNT 2007 
\bibitem{genetics}  Riccardo Poli, William B. Langdon, Nicholas F. McPhee, John R. Koza: A Field Guide to Genetic Programming. Lulu Enterprises Uk Ltd 2008
\bibitem{SPICE} https://mfiles.pl/pl/index.php/Automotive\_SPICE
\end{thebibliography}

\clearpage

\makesummary

\end{document} 
